\documentclass[a4paper, notitlepage]{article}
%\usepackage[cm]{fullpage}
\usepackage[pdftex]{graphicx}

\begin{document}

\title{Installation and User Guide} 
\date{\today}
\maketitle



\section{Requirements}

\begin{itemize}
	\item Internet connection
	\item The following tools must be installed (lower versions may work, but not recommended)
	\begin{itemize}
		\item \textit{Java SDK} ($>=$ 7)
		\item \textit{Git} - to get the code - http://git-scm.com/downloads
		\item \textit{Maven} ($>=$ 3) - for building - http://maven.apache.org/download.html
		\item \textit{Apache Tomcat} ($>=$ 8) - for deploying the web interface - http://tomcat.apache.org/
	\end{itemize}
\end{itemize}

\section{Get the code}
Using Git clone the repository at GitHub using the following command:

\textit{git clone https://github.com/btodorov88/experiments.git}

\section{Build the project}

In order to build the project the following steps have to be executed using \textit{CMD} or alternative:
\begin{enumerate}
	\item Navigate to the SpringBankWeb in the cloned repository
	\item Execute the \textit{mvn clean package} command
\end{enumerate}

\section{Deploy}
The build procedure creates deployable \textit{war} file. It can be found at \textit{SpringBankWeb\textbackslash target\textbackslash springbank-1.0.0-BUILD-SNAPSHOT}. In order to deploy the file, one has to copy it to the \textit{webapps} folder of the Tomcat web server(for easier access the file should be renamed to \textit{springbank.war}). Once deployed the web client can be accessed at \textit{localhost:8080\textbackslash springbank} (default Tomcat configuration).


\section{Explore the application}

When the application is started it redirects to a login page for authentication. The test data defines a user \textit{btodorov} with password \textit{123}. Once successfully authenticated the user is redirected to the main page which presents all accounts (left table) and transactions (right table) associated with the authenticated user. 

A new bank transaction can be made by clicking on the "Create transaction" button in the menu. Next the user is asked to fill in a form providing details about the transaction. When the form is submitted it is validated to make sure all necessary fields are populated and the "From account" has enough capital. In case of success the user is redirected to the main page where the newly created transaction can be seen. 

\section{Take a look at the code}
The cloned project can be easily imported into Eclipse. In order to do that go to \textit{File\textbackslash Import \textbackslash Existing Projects into Workspace} and follow the instructions.

\end{document}